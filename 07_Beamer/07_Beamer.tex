\documentclass[11pt]{beamer}
\usetheme{Szeged}
\usepackage[utf8]{inputenc}
\usepackage[magyar]{babel}
\usepackage[T1]{fontenc}
\usepackage{graphicx}
\usepackage{verbatim}
\usepackage{hyperref}

\author{Tipográfiai rendszerek - \TeX}
\title{Beamer}

\date{2021.04.14.} 

\newcommand{\tbs}{\textbackslash}

\begin{document}

\begin{frame}
\titlepage
\end{frame}

\begin{frame}{Mi az a beamer?}
\begin{itemize}
\item Ehhez a \textbf{beamer} dokumentumosztályt kell használnunk
	\begin{itemize}
	\item \tbs documentclass\{beamer\}
	\end{itemize}
\item A beamer a \TeX{} \tbs \  \LaTeX{} \tbs \  ...\TeX{} prezentációkészítője
\item Előnye a ,,hagyományos'' prezentációkészítőkkel szemben ugyanaz, mint a többi dokumentumosztály esetén
	\begin{itemize}
	\item platformfüggetlen
	\item garantált azonos megjelenés minden felületen
	\item tipográfiai \tbs \ megjelenítési harmóniára való törekedés
	\end{itemize}
\end{itemize}
\end{frame}

\begin{frame}{Témák}
\begin{itemize}
\item A \textbf{\tbs usetheme\{témaneve\}} módon választhatunk témát
\item Az alapból definiált témák városok neveiről vannak elnevezve
\item A témákhoz általában tartoznak színvariánsok is
\item Ezek pedig többségében állatokról vannak elnevezve
\item Ezt pedig a \textbf{\tbs usethemecolor\{színvariáns neve\}} módon használhatjuk
\item Ezeket lásd pl. \url{https://hartwork.org/beamer-theme-matrix/}
\item De, természetesen, ha nem tetszenek, vagy sajátot szeretnénk, ezeket is újra lehet definiálni, vagy egy teljesen újat készíteni.
\end{itemize}
\end{frame}

\begin{frame}{A preambulum fontosabb részei}
\begin{itemize}
\item \tbs title\{A prezentáció címe\}
\item \tbs author\{Vezetéknév Keresztnév\}
\item \tbs institute\{Példa Statisztikai Intézmény\}
\item \tbs date\{ÉÉÉÉ.HH.NN.\}
\end{itemize}
\end{frame}

\begin{frame}{A diák készítése}
\begin{itemize}
\item A diák a \textbf{frame} környezetben vannak definiálva
	\begin{itemize}
	\item \tbs begin\{frame\}
	\item A dia tartalma
	\item \tbs end\{frame\}
	\end{itemize}
\item A dia címét vagy a \textbf{\tbs frametitle\{A dia címe\}} paranccsal adhatjuk meg
\item vagy a \tbs begin\{frame\}\textbf{\{Dia címe\}}  módon
\end{itemize}
\end{frame}

\begin{frame}[fragile]{Az első dia}
\begin{itemize}
\item Az ajánlott első dia a címdia
\item Ezt automatikusan hozzáadja a varázsló
\item Ezen jelennek meg a preambulumban deklarált értékek
\end{itemize}

\vspace{1cm}

\verb|\begin{frame}| \\
\verb|\titlepage| \\
\verb|\end{frame}|
\end{frame}

\begin{frame}{Strukturalizáció - opcionális}
\begin{itemize}
\item A prezentációt lehetőségünk van strukturalizálni
\item A nagy részeket a \textbf{\tbs part\{Rész neve\}} választja el - általában nem jelenik meg sehol
	\begin{itemize}
	\item általában nem jelenik meg a címe
	\item célja a részek számozásának elválasztása
	\end{itemize}
\item A részeken belüli szekciókat / fejezeteket a \textbf{\tbs section\{Fejezet címe\}} módon adhatjuk meg
	\begin{itemize}
	\item Ezek neve a téma beállításától függő helyen - általában a fejlécben - megjelenik
	\item Valamint listázásra kerül a diák mennyisége is, hivatkozott formában
	\end{itemize}
\end{itemize}
\end{frame}

\begin{frame}[fragile]{Érdekesség}
\begin{itemize}
\item Amennyiben egy felsorolást szeretnénk készíteni, aminek az elemei egyesével jelennek meg, úgy azt az alábbi módon tehetjük meg:
\end{itemize}

\begin{verbatim}
\begin{itemize}
 \item<1-> Első diától jelenik meg
 \item<2-> Második diától jelenik meg
 \item<3-> Harmadik diától jelenik meg
 \item<4-> Negyedik diától jelenik meg
\end{itemize}
\end{verbatim}

\begin{itemize}
\item Ha azt szeretnénk, hogy valamelyik csak az adott dián - se előtte, se utána - jelenjen meg
\item akkor ott hagyjuk el a ,,-''-et a szám után, pl. ,,<3>''
\end{itemize}
\end{frame}

\begin{frame}
\frametitle{A gyakorlatban}
\begin{itemize}
 \item<1-> Első diától jelenik meg
 \item<2-> Második diától jelenik meg
 \item<3-> Harmadik diától jelenik meg
 \item<4-> Negyedik diától jelenik meg
\end{itemize}
\end{frame}

\begin{frame}
\frametitle{A gyakorlatban a kihagyásos}
\begin{itemize}
 \item<1-> Első diától jelenik meg
 \item<2-> Második diától jelenik meg
 \item<3> \textbf{Csak} a harmadik dián jelenik meg
 \item<4-> Negyedik diától jelenik meg
\end{itemize}
\end{frame}

\begin{frame}[fragile]{Másik lehetőség}
\begin{itemize}
\item Amikor nem felsorolással dolgozunk, akkor használhatjuk a \textbf{\tbs pause} parancsot is
\end{itemize}

\begin{verbatim}
\begin{frame}

Első dia szövege \\ \pause % Ha, nem adunk sortörést, 
							akkor egy sorba írja
Második dia szövege \\ \pause
Harmadik dia szövege
\end{verbatim}
\verb|\end{frame}|
\end{frame}

\begin{frame}
\frametitle{Gyakorlatban a pause}
Első dia szövege \\ \pause
Második dia szövege \\ \pause
Harmadik dia szövege
\end{frame}

\end{document}
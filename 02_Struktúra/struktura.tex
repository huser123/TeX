\documentclass[11pt]{beamer}
 \usetheme{Szeged}
 \usepackage[utf8]{inputenc}
 \usepackage[magyar]{babel}
 \usepackage[T1]{fontenc}
%\usepackage{booktabs} 

 \author{Tipográfiai rendszerek - \TeX}
 \title{Formázás és struktúrák}
 \date{2021.02.24.} 


\newcommand{\tbs}{\textbackslash} 

 
 \begin{document}
 
 \begin{frame}
 \titlepage
 \end{frame}
 
\begin{frame}{Ami kell:}

\textbf{\tbs documentclass}[10pt,a4paper]\{book\} \\
\textbf{\tbs usepackage}[utf8]\{inputenc\} \\
\textbf{\tbs usepackage}[magyar]\{babel\} \\
\textbf{\tbs usepackage}[T1]\{fontenc\} \\
\textbf{\tbs author}\{Vezetéknév Keresztnév\} \\
\textbf{\tbs title}\{Formázás és struktúra\}

\vspace{1cm}

\textbf{\tbs begin}\{document\} \\
\hspace{1cm} tartalom \\
\textbf{\tbs end}\{document\}

\end{frame}
 
 
\begin{frame}{Új sor}
\begin{itemize}
 
\item \tbs newline - új sor, \textbf{de nem új bekezdés} 
\item \tbs \tbs \ - új sor, \textbf{de nem új bekezdés}
\item \tbs vspace\{...cm\} - adott egységnyi szabad függőleges helykihagyása
\item \tbs hspace\{...cm\} - adott egységnyi szabad vízszintes helykihagyása
\item De használhatjuk így is: \tbs \tbs [\dots pt] \% vagy más mértékegység
\end{itemize} 
\end{frame}

\begin{frame}{Szövegformázás}
\begin{itemize}
\item	\tbs textbf\{•\} - \textbf{félkövér}
\item 	\tbs textit\{•\} - \textit{dőlt}
\item   \tbs emph\{•\} - \emph{kiemelt} - csomagonként változhat az stílusa!
\item 	\tbs texttt\{•\} - \texttt{írógép}
\item   \tbs underline\{•\} - \underline{aláhúzott}
\end{itemize}
\end{frame}

\begin{frame}[fragile=singleslide]{Idézőjelek}
\begin{itemize}

\item	A magyarban az idézőjeleket alul kezdjük és felül zárjuk
\item 	szemben az angollal, ami csak felül jelöli "-el
\item	A magyar idézőjelet  \verb+,,+ -el kezdjük és \verb+''+ -el zárjuk:
		\begin{itemize}
		\item ,,Az ég kék''
		\end{itemize}
\end{itemize}
\end{frame}

\begin{frame}[fragile=singleslide]{Kötőjelek és a három pont}
\begin{itemize}
\item	- (\verb+-+) - kiskötőjel 
\item	-- (\verb+--+) - gondolatjel
\item 	--- (\verb+---+) - kvirtmínusz
\item	\tbs dots - \dots
\end{itemize}
\end{frame}

\begin{frame}{Elválasztás}
\begin{itemize}
\item	A \TeX \ a babel csomag alapján ismeri a nyelv szabályait, így az elválasztást is.
\item	De,előfordulhatnak speciális esetek, vagy más, amikor nekünk kell jelezni, hogy egy szó, hol választható el.
\item	Ekkor a szóba illesztett \tbs - -el tudjuk jelezni a \TeX \ számára az elválasztási helyet
\item	Például:
		\begin{itemize}
		\item megszentségteleníthetetlenségeskedéseitekért
		\item meg\tbs -szent\tbs -ség\tbs -te\tbs -le\tbs -nít\tbs -he\tbs -tet\tbs -len\tbs -sé\tbs -ges\tbs -ke\tbs -dé\tbs -se\tbs -i\tbs -te\tbs -kért
		\end{itemize}
\end{itemize}
\end{frame}

\begin{frame}{Új oldal és oldalszámozás}
\begin{itemize}
\item \tbs newpage
\item \tbs pagenumbering\{\}
	\begin{itemize}
	\item arabic
	\item roman
	\item Roman
	\item alph
	\item Alph
	\end{itemize}
\item \tbs pagestyle\{empty\} \% A preambulumba kell írni - ekkor egyik oldal sem lesz számozva
\item \tbs thispagestyle\{empty\} \% Az adott laphoz, amelyiken nem szeretnénk az oldalszámot
\item \tbs setcounter\{page\}\{szam\} \% Átállítja az oldalszámozást
\end{itemize}
\end{frame}

\begin{frame}{Szakaszolás}
\begin{itemize}
\item	\tbs part\{cím\} - rész (csak book!)
\item	\tbs chapter\{cím\} - fejezet (csak book!)
\item	\tbs section\{cím\} - szakasz / fejezet
\item   \tbs subsection\{cím\} - alszakasz / alfejezet
\item	\tbs subsubsection\{cím\} - alalszakasz / alalfejezet
		\begin{itemize}
		\item	nincs számozva
		\item	a tartalomjegyzékben sem jelenik meg
		\end{itemize}
\item	\tbs appendix\{cím\} - függelékek - betűvel jelenik meg szám helyett
\end{itemize}
\end{frame}
 
\begin{frame}{Számozatlan szakaszolás}
\begin{itemize}
\item A csillaggal jelölt tagolók nem számozódnak és a tartalomjegyzékben sem jelennek meg
\end{itemize}

\vspace{1cm}

\begin{itemize}
\item	\tbs part*\{cím\}
\item	\tbs chapter*\{cím\}
\item	\tbs section*\{cím\}
\item   \tbs subsection*\{cím\}
\item	\tbs subsubsection*\{cím\}
\end{itemize}
\end{frame}

\begin{frame}{Tartalomjegyzék}
\begin{itemize}
\item	\tbs tableofcontents
\item	Többször kell fordítani a frissítéséhez
\end{itemize}
\end{frame}


\begin{frame}{Margók}
\begin{itemize}
\item A margókat a \textbf{geometry} csomaggal szabályozhatjuk
\item \tbs usepackage[left=\dots cm,right=\dots cm,top=\dots cm,
bottom=\dots cm]\{geometry\}
\end{itemize}
\end{frame}

% Dokumentáció a fancyhdr doksiban: https://mirror.szerverem.hu/ctan/macros/latex/contrib/fancyhdr/fancyhdr.pdf

\begin{frame}{Fejléc és lábléc}
\begin{itemize}
\item Ehhez új csomagra lesz szükségünk
\item \tbs usepackage\{fancyhdr\}
\item alkalmazni kell a csomagot: \tbs pagestyle\{fancy\}
\item a preambulumba írjuk be:
	\begin{itemize}
	\item \tbs pagestyle\{fancy\}
	\item \tbs fancyhf\{\} \% törli az addigi fej- és lábléc elemeket
	\item \tbs lhead\{baloldali fejléc mező\}
	\item \tbs chead\{középső fejléc mező\}
	\item \tbs rhead\{jobboldali  fejléc mező\}
	\item 
	\item \tbs lfoot\{baloldali lábléc mező\}
	\item \tbs cfoot\{középső lábléc mező\} \% pl. \tbs thepage - oldalszám
	\item \tbs rfoot\{jobboldali  lábléc mező\}
	\end{itemize}
\end{itemize}
\end{frame}

\begin{frame}{Fejléc és lábléc}
\begin{itemize}
\item Az elválasztó vonalak vastagságát az alábbiakkal adjuk meg:
	\begin{itemize}
	\item \tbs renewcommand\{\tbs headrulewidth\}\{\dots pt\}
	\item \tbs renewcommand\{\tbs footrulewidth\}\{\dots pt\}
	\end{itemize}
\item Egyéb elemek:
	\begin{itemize}
	\item \tbs thepage \% Az adott oldal oldalszáma
	\item \tbs thechapter \% Az fejezet \textbf{számát} adja vissza
	\item \tbs thesection \% Az adott ,,section'' \textbf{számát} adja vissza
	\item \tbs chaptername\% kiírja, hogy fejezet
	\item \tbs leftmark \% Kiírja a legmagasabb struktúra számát és nevét - pl. FEJEZET 1. A FEJEZET CÍME
	\item \tbs rightmark \% Kiírja a második legmagasabb struktúra számát és nevét - pl. 1.1 EZ A SECTION CÍME
	\end{itemize}
\end{itemize}
\end{frame}



% Lásd: https://www.overleaf.com/learn/latex/headers_and_footers

\begin{frame}{Fejléc és lábléc}
\begin{itemize}
\item Természetesen, ezeket is lehet még tovább finomítani (nem minden class esetén működik!)
\item \tbs fancyhead$[$LE,RO$]$\{szöveg\}
	\begin{itemize}
	\item E - páros oldal (even page)
	\item O - páratlan (odd page)
	\item L - baloldal (left side)
	\item C - középen (centered)
	\item R - jobboldal (right side)
	\end{itemize}
\end{itemize}
\end{frame}

\end{document} 

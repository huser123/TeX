\documentclass[11pt]{beamer}
\usetheme{Szeged}
\usepackage[utf8]{inputenc}
\usepackage[magyar]{babel}
\usepackage[T1]{fontenc}
\usepackage{amsmath}
\usepackage{amsfonts}
\usepackage{amssymb}
\usepackage{graphicx}
\usepackage{tcolorbox}
\usepackage{multicol}

\author{Tipográfiai rendszerek - \TeX}
\title{Matematikai mód}
\date{2021.03.10.}

\newcommand{\tbs}{\textbackslash}
 
\begin{document}

\begin{frame}
\titlepage
\end{frame}

\begin{frame}{Mi is az a matematikai mód?}
\begin{itemize}
\item 	Ehhez szükségünk van az ams csomagokra!
\item	A \TeX{} (és így a \LaTeX{}) fő erőssége a matematikai szövegek szedése
\item	A matematikai szövegek szedéséhez ún. ,,matematikai módot'' használunk
\item	A szövegközi matematikai mód két \textbf{\$ képlet \$} közé íródik.
\item 	A kiemelt matematikai mód vagy \textbf{\$\$ képlet \$\$} közé, vagy pedig \textbf{\tbs $\left[ \right.$ képlet \tbs $\left. \right]$} közé.
\item 	A sortöréseknek és a szóközöknek nincs értelme, ezeket parancsokkal kell meghatározni
	\begin{itemize}
	\item pl. \textbf{\tbs quad}
	\end{itemize}
\end{itemize}
\end{frame}

\begin{frame}[fragile]{Matematikai mód - példák}

Ez a képlet $c^2=a^2+b^2$ a szövegben fog elhelyezkedni. \\

De kiemelt matematikai módban $$c^2=a^2+b^2$$ már nem.

\vspace{1cm}

\begin{verbatim}
Ez a képlet $c^2=a^2+b^2$ a szövegben fog elhelyezkedni. \\

De kiemelt matematikai módban $$c^2=a^2+b^2$$ már nem.
\end{verbatim}
\end{frame}

\begin{frame}{Váltás módok közt}
\begin{itemize}
\item Gyakran szükség lehet, hogy matematikai módból visszaváltsunk szöveges módba
\item Például szövegek beírásakor
\item Mert a \TeX{} a matematikai módban írt betűket ismeretleneknek / változóknak veszi.
\item Erre szolgál a matematikai módban használható \textbf{\tbs textrm\{szöveg\}} parancs.
\end{itemize}
\end{frame}

\begin{frame}[fragile]{Használat}
\begin{itemize}
\item Felső index: $\hat{}$ pl: a \verb+a^2+ ($a^2$)
\item Alsó index: \_ pl. \verb+a_2+ ($a_2$)
\item Az alsó és a felső index egyszerre csak egy karakterre vonatkozik!
\item azaz a \verb|a^2365|-ból $a^2365$ lesz, a \verb|a_2365|-ból $a_2365$
\item ilyenkor csoportosítani kell a \{\} közé:
	\begin{itemize}
	\item \verb|a^{2365}| $\rightarrow$ $a^{2365}$
	\item \verb|a_{2365}| $\rightarrow$ $a_{2365}$
	\end{itemize}
\item kiemelendő még az alábbi sajátosság is:
\item \verb|a^2_3| az $a^2_3$ de a \verb|a_3^2| is $a_3^2$
\end{itemize}
\end{frame}

\begin{frame}[fragile]{A hatvány hatványa, az index indexe, ...}
\begin{itemize}
\item Ilyenkor \{\} közt kell megadni az adatokat
\item $a^{b^c}$ esetén \verb|a^{b^c}|
\item $a_{b_c}$ esetén \verb|a_{b_c}|
\item $a^{b_c}$ esetén \verb|a^{b_c}|
\item $a_{b^c}$ esetén \verb|a_{b^c}|
\item $a^{b+c}$ esetén \verb|a^{b+c}|
\item $a_{b+c}$ esetén \verb|a_{b+c}|
\end{itemize}
\end{frame}

\begin{frame}[fragile]{Gyök}
\begin{itemize}
\item Gyökvonást a \textbf{\tbs sqrt\{szám\}} paranccsal készítünk.
\item A gyökjel nagyságát a \TeX{} határozza meg.
\item Pl. $\sqrt{5646546546454}$ azaz \verb|\sqrt{5646546546454}|
\item x-edik gyökre a \verb|\sqrt[•]{•}| parancs használható.
\item Pl. $\sqrt[3]{46456466}$ azaz \verb|\sqrt[3]{46456466}|
\end{itemize}
\end{frame}

\begin{frame}[fragile]{Törtek}
\begin{itemize}
\item A törtek készítésére a \textbf{\tbs frac\{•\}\{•\}} parancs való.
\item Pl. $\frac{54645}{21315}$ azaz \verb|\frac{54645}{21315}|
\item A törtek egymásba ágyazhatóak, azaz emeletes törteket hozhatunk létre.
\item $$\frac{\frac{a}{b}}{\frac{c}{d}}$$ azaz \verb|\frac{\frac{a}{b}}{\frac{c}{d}}|
\item De, szabadon kombinálható mással is: $$\frac{\sqrt[a^2]{2}}{\sqrt[b_i]{3}}$$ azaz \verb|\frac{\sqrt[a^2]{2}}{\sqrt[b_i]{3}}|
\end{itemize}
\end{frame}

\begin{frame}{Matematikai szimbólumok}
\begin{itemize}
\item A halmazokat dupla szárú betűvel jelöljük.
\item ezeket a \textbf{\tbs mathbb\{•\}} paranccsal tudjuk kiíratni.
\item például:
	\begin{itemize}
	\item a természetes számok halmaza $\mathbb{N}$
	\item az egész számok halmaza $\mathbb{Z}$
	\item a racionális számok halmaza $\mathbb{Q}$
	\item a valós számok halmaza $\mathbb{R}$
	\item a komplex számok halmaz $\mathbb{C}$
	\end{itemize}
\end{itemize}
\end{frame}

\begin{frame}{Matematikai szimbólumok}
\begin{itemize}
\item A \TeX{} a görög betűket is képes kiírni.
\item \tbs pi $\pi$
\item \tbs Pi $\Pi$
\item \tbs alpha $\alpha$
\item \tbs beta $\beta$
\item \dots
\end{itemize}
\end{frame}

\begin{frame}[fragile]{Matematikai szimbólumok}
\begin{itemize}
\item Ezen kívül a \TeX{} számtalan az alap és a magasabb szintű matematikához szükséges szimbólum kezelésére is képes. Ezek megtekinthetőek a különböző útmutatókban, illetve az editor különböző menüpontjaiban.
\item Példák:
	\begin{itemize}
	\item	\verb|\lim_{n \to \infty}\frac{1}{n}|
	\item 	\verb|T = \int^b_a f(x) \ dx|
	\item 	\verb|a \in \mathbb{R}^+|
	\end{itemize}
\end{itemize}

$$\lim_{n \to \infty}\frac{1}{n}$$

$$T = \int^b_a f(x) \ dx$$

$$a \in \mathbb{R}^+$$

\end{frame}

\begin{frame}[fragile]{Számozott egyenletek}
\begin{itemize}
\item A számozott egyenleteket a\\ \textbf{\tbs begin\{equation\}} és \textbf{\tbs end\{equation\}} közé írjuk
\end{itemize}

\begin{equation}
c^2=a^2+b^2
\end{equation}

\begin{equation}
a=\sqrt{c^2-b^2}
\end{equation}

\begin{verbatim}
\begin{equation}
c^2=a^2+b^2
\end{equation}

\begin{equation}
a=\sqrt{c^2-b^2}
\end{equation}
\end{verbatim}
\end{frame}

\begin{frame}[fragile]{,,Egyenlettömb''}
\begin{itemize}
\item Egyenlettömböt az \textbf{eqnarray} környezetben hozhatunk létre
\item Minden új sora egy új számozott képlet / egyenlet
\end{itemize}
\begin{eqnarray}
a+b=c \\
x+y=z \\
c^2=a^2+b^2
\end{eqnarray}
\begin{verbatim}
\begin{eqnarray}
a+b=c \\
x+y=z \\
c^2=a^2+b^2
\end{eqnarray}
\end{verbatim}
\end{frame}

\begin{frame}[fragile]{Rendezett egyenletek}
\begin{itemize}
\item Az egyenleteket rendezhetjük egymás alá, mondjuk egyenlőségjellel
\item itt az \textbf{align} környezetet használjuk ehhez
\item A rendezési pontot az \textbf{\&} jellel adjuk meg.
\end{itemize}

\begin{align}
aa+ab&=cc \\
x+y&=z+n \\
c^2&=a^2+b^2
\end{align}

\begin{verbatim}
\begin{align}
aa+ab&=cc \\
x+y&=z+n \\
c^2&=a^2+b^2
\end{align}
\end{verbatim}
\end{frame}

\begin{frame}{Hivatkozások}
\begin{itemize}
\item Természetesen, ezekre hivatkozni is tudunk
\item Hivatkozáshoz ugyanúgy a \textbf{\tbs label\{címke-neve\}} parancsot használjuk
\item A hivatkozás helyén pedig a \textbf{\tbs ref\{cimke-neve\}} parancsot
\end{itemize}
\end{frame}

\begin{frame}{Számozatlan környezetek}
\begin{itemize}
\item Természetesen, ezeknek a környezeteknek is vannak számozatlan változataik
\item Ebben az esetben a begint és a \{ közé csillagot kell rakni
\item Például:
	\begin{itemize}
	\item \tbs begin*\{equation\}
	\item \tbs begin*\{eqnarray\}
	\item \tbs begin*\{array\}\{•\}
	\end{itemize}
\end{itemize}
\end{frame}

\begin{frame}{Zárójelek}
\begin{itemize}
\item A \TeX{} képes a zárójelek nagyságát automatikusan igazítani.
\item Ehhez a zárójeleket parancsokkal kell megadni:
	\begin{itemize}
	\item \tbs left(\dots\tbs right) $\rightarrow$ $\left(\dots\right)$
	\item \tbs left[ \dots \tbs right] $\rightarrow$ $\left[\dots\right]$
	\item \tbs left\tbs lbrace \dots \tbs right\tbs rbrace $\rightarrow$ $\left\lbrace\dots\right\rbrace$
	\item \tbs left\tbs langle\dots\tbs right\tbs rangle $\rightarrow$ $\left\langle\dots\right\rangle$
	\item \tbs left|\dots\tbs right| $\rightarrow$  $\left|\dots\right|$
	\end{itemize}
\item A zárójeleknek mindig kell nyitó és záró tag.
\item Amennyiben valamelyik tagra nincs szükségünk, úgy azt a "csendesnek" kell jelölnünk $\rightarrow$ \tbs left. vagy \tbs right.
	\begin{itemize}
	\item tehát a left, vagy a right után egy pontot teszünk.
	\end{itemize}
\end{itemize}
\end{frame}

\begin{frame}{Zárójel igazítás a gyakorlatban}
\begin{Huge}
$$ \left[\sqrt[\frac{a^{3_k}}{k_{p^c}}]{\frac{\sqrt[a^c]{\frac{\sqrt[c_j]{p^{d^c}}}{a_{k+c^p}}}}{m_{pl}+c_l}}+\left(\frac{\frac{\sqrt[q^4]{m_l}}{a^{p+c_k}+c_d}}{\frac{\sqrt[x_p]{a^{p+c}}}{k_{l_m}+p^{q^r}}}\right)\right]$$
\end{Huge}
\end{frame}

\begin{frame}[fragile]
$$S_n=\underbrace{a_1+a_2+\cdots+a_n}_{n \textrm{ db}}=\sum_{i=1}^n a_i$$
\vspace{1cm}
\begin{verbatim}
$$S_n=\underbrace{a_1+a_2+\cdots+a_n}_{n \textrm{ db}}=
\sum_{i=1}^n a_i$$
\end{verbatim}
\end{frame}

\begin{frame}[fragile]{Tömb}
\begin{itemize}
\item Ez az \textbf{array} környezet
\item szintaxisa megegyezik a táblázatokéval
\item mátrixok létrehozására (is) használható
\end{itemize}

\begin{multicols}{2}
$$A=\left( \begin{array}{ccc}
1 & 2 & 3 \\ 4 & 5 & 6 \\ 7 & 8 & 9
\end{array}   \right)$$

$$\det A=\left\vert \begin{array}{ccc}
1 & 2 & 3 \\ 4 & 5 & 6 \\ 7 & 8 & 9
\end{array}   \right\vert$$
\end{multicols}

\begin{verbatim}
$$A=\left( \begin{array}{ccc}
1 & 2 & 3 \\ 4 & 5 & 6 \\ 7 & 8 & 9
\end{array}   \right)$$
$$\det A=\left\vert \begin{array}{ccc}
1 & 2 & 3 \\ 4 & 5 & 6 \\ 7 & 8 & 9
\end{array}   \right\vert$$
\end{verbatim}
\end{frame}

\begin{frame}[fragile]{m,n elemű tömb}

$$A_{m,n} = \left(
\begin{array}{cccc}
a_{1,1} & a_{1,2} & \cdots & a_{1,n} \\
a_{2,1} & a_{2,2} & \cdots & a_{2,n} \\
\vdots  & \vdots  & \ddots & \vdots  \\
a_{m,1} & a_{m,2} & \cdots & a_{m,n} 
\end{array} \right)$$

\begin{verbatim}
$$A_{m,n} = \left(
\begin{array}{cccc}
a_{1,1} & a_{1,2} & \cdots & a_{1,n} \\
a_{2,1} & a_{2,2} & \cdots & a_{2,n} \\
\vdots  & \vdots  & \ddots & \vdots  \\
a_{m,1} & a_{m,2} & \cdots & a_{m,n} 
\end{array} \right)$$
\end{verbatim}
\end{frame}

\begin{frame}{Továbbiak}
\begin{itemize}
\item Az itt bemutatott matematikai képességek a teljesség igénye nélkül készültek
\item A cél a mindennapokban többször használt dolgok és az azokhoz használt logika bemutatása volt
\item A \TeX{} ezen kívül számtalan matematikához kapcsolható szerkesztésre képes
\item Szinte mindenre létezik csomag és hozzá leírás
\end{itemize}
\end{frame}

\end{document}
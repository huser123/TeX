\documentclass[11pt]{beamer}
\usetheme{Szeged}
\usepackage[utf8]{inputenc}
\usepackage[magyar]{babel}
\usepackage[T1]{fontenc}
\usepackage{graphicx}

\usepackage{biblatex}

\author{Tipográfiai rendszerek - \TeX
}
\title{Bib(la)tex, saját dokumentum osztályok}

\date{2021.03.31.} 

\newcommand{\tbs}{\textbackslash}

\begin{document}

\begin{frame}
\titlepage
\end{frame}

%\begin{frame}
%\tableofcontents
%\end{frame}

\begin{frame}{Bibliográfiai hivatkozások}
\begin{itemize}
\item A \TeX{}-ben lehetőségünk van az irodalmi hivatkozásokat rendszerezni
\item Ennek egyik megoldása a Biblatex
\item Ehhez előbb létre kell hozni egy .bib kiterjesztésű adatbázist
\item Ebbe gyűjteni az irodalmakat
\item Ehhez szükség van a \textbf{biblatex} csomagra
\item A .bib fájlt pedig a \textbf{\tbs addbibresource\{bibfajl\}}-al hivatkozzuk le
	\begin{itemize}
	\item és a Biber fordítóra, ami egy külön szoftver
	\end{itemize}
\end{itemize}
\end{frame}

\begin{frame}{.bib fájl}
\begin{itemize}
\item A bib fájl tartalmazza az irodalom minden elemét
	\begin{itemize}
	\item címke - amivel hivatkozunk rá
	\item A mű címe
	\item A mű szerzője
	\item A mű szerzői
	\item Kiadó
	\item év
	\item \dots
	\end{itemize}
\end{itemize}
\end{frame}

\begin{frame}[fragile]{Példa}
\begin{verbatim}
@article{einstein,
    author = "Albert Einstein",
    title = "{Zur Elektrodynamik bewegter K{\"o}rper}.
     ({German})
    [{On} the electrodynamics of moving bodies]",
    journal = "Annalen der Physik",
    volume = "322",
    number = "10",
    pages = "891--921",
    year = "1905",
    DOI = "http://dx.doi.org/10.1002/andp.19053221004",
    keywords = "physics"
}
\end{verbatim}
\end{frame}

\begin{frame}
\begin{itemize}
\item Ezekre hivatkozni a \textbf{\tbs cite\{cimke\}} paranccsaltudunk
\item Természetesen, a parancs újra kalibrálásával egyedi formát is adhatunk neki
\item a végén pedig a \textbf{\tbs printbibliography} paranccsal listázhatjuk ki a hivatkozott irodalmakat.
\end{itemize}
\end{frame}

\begin{frame}[fragile]
\begin{itemize}
\item A TeXMakerben érdemes átállítani a Bib(la)tex fordítót biberre
	\begin{itemize}
	\item biber \%
	\end{itemize}
\item Esetleg saját parancsot létrehozni
	\begin{itemize}
	\item \begin{verbatim}
	pdflatex -synctex=1 -interaction=nonstopmode %.tex|
	biber % | pdflatex %|pdflatex -synctex=1 
	-interaction=nonstopmode %.tex|pdflatex -synctex=1 
	-interaction=nonstopmode %.tex|xdg-open %.pdf
	\end{verbatim}
	\end{itemize}
\item \textbf{Csak olyan, aki tudja, hogy mit csinál!}
	\begin{itemize}
	\item A fentebbi parancsok Linux alatt futnak biztosan
	\item A Windows verziók eltérhetnek
	\end{itemize}
\end{itemize}
\end{frame}

\begin{frame}
\includegraphics[scale=0.5]{kep} 
\end{frame}

\begin{frame}{Saját dokumentum osztály}
\begin{itemize}
\item A TeX-ben lehetőségünk van létrehozni saját dokumentum osztályokat is
\item Ehhez egy .cls fájlt kell létrehoznunk
\end{itemize}
\end{frame}

\begin{frame}{A cls fájl felépítése - \textbf{nem} teljes}
\begin{itemize}
	\item \tbs NeedsTeXFormat\{LaTeX2e\} \% Használni kívánt formátum
	\item \tbs ProvidesClass\{exampleclass\}[2014/08/16 Example LaTeX class] \% Az osztály neve
	\item 
	\item \tbs LoadClass\{osztaly\} \% Az osztály, amire épül
	\item \tbs RequirePackage\{csomag\_neve\} \% A megkövetelt csomagok - ezeket  külön már nem kell betölteni
	\item \tbs newcommand\{\tbs uj\_parancs\}\{\tbs regi\_parancs\} \% Új parancsok definiálása
\end{itemize}
\end{frame}

\begin{frame}{Minta}

\tbs NeedsTeXFormat\{LaTeX2e\} \\
\tbs ProvidesClass\{sajat\}[2021/03/31 Bemutato osztaly] \\
\ \\
\tbs LoadClass[10pt,a4paper]\{article\} \\
\tbs RequirePackage[utf8]\{inputenc\} \\
\tbs RequirePackage[magyar]\{babel\} \\
\tbs RequirePackage[T1]\{fontenc\} \\
\tbs RequirePackage\{graphicx\} \\
\tbs RequirePackage[backend=biber]\{biblatex\} \\
\tbs RequirePackage\{color\} \\
\tbs RequirePackage[left=2cm,right=2cm,top=2cm,bottom=2cm]\{geometry\} \\
\tbs RequirePackage\{xcolor\} \\
\tbs RequirePackage\{amsmath\} \\
\tbs RequirePackage\{amsfonts\} \\
\end{frame}

\begin{frame}
\tbs RequirePackage\{amssymb\} \\
\tbs RequirePackage\{sectsty\} \\
\ \\
\tbs definecolor\{foszin\}\{RGB\}\{100,255,140\} \\
\ \\
\tbs sectionfont\{\tbs color\{foszin\}\} \\
\ \\
\tbs newcommand\{\tbs tbs\}\{\tbs textbackslash\} \\
\tbs newcommand\{\tbs fejezet\}\{\tbs section\} \\
\ \\
\tbs def\tbs C\{\tbs mathbb\{C\}\}
\end{frame}

\end{document}